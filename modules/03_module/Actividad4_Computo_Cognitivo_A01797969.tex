\documentclass[12pt,a4paper]{article}

% Paquetes
\usepackage[utf8]{inputenc}
\usepackage[spanish]{babel}
\usepackage{graphicx}
\usepackage{hyperref}
\usepackage{geometry}
\usepackage{fancyhdr}
\usepackage{listings}
\usepackage{xcolor}
\usepackage{longtable}
\usepackage{booktabs}

% Configuración de página
\geometry{margin=2.2cm}
\pagestyle{fancy}
\fancyhf{}
\rhead{A01797969}
\lhead{Actividad 4 - Cómputo Cognitivo}
\rfoot{\thepage}

% Configuración de código
\definecolor{codegreen}{rgb}{0,0.6,0}
\definecolor{codegray}{rgb}{0.5,0.5,0.5}
\definecolor{codepurple}{rgb}{0.58,0,0.82}
\definecolor{backcolour}{rgb}{0.95,0.95,0.92}

\lstdefinestyle{mystyle}{
    backgroundcolor=\color{backcolour},   
    commentstyle=\color{codegreen},
    keywordstyle=\color{blue},
    numberstyle=\tiny\color{codegray},
    stringstyle=\color{codepurple},
    basicstyle=\ttfamily\footnotesize,
    breakatwhitespace=false,         
    breaklines=true,                 
    captionpos=b,                    
    keepspaces=true,                 
    numbers=left,                    
    numbersep=5pt,                  
    showspaces=false,                
    showstringspaces=false,
    showtabs=false,                  
    tabsize=2
}
\lstset{style=mystyle}

% Compactar listas
\usepackage{enumitem}
\setlist{itemsep=1pt, parsep=1pt, topsep=3pt}

\begin{document}

% Portada
\begin{titlepage}
    \centering
    \vspace*{2cm}
    {\LARGE\bfseries Actividad 4\par}
    \vspace{0.5cm}
    {\Large Exploración de Herramientas de Cómputo Cognitivo\par}
    \vspace{2cm}
    {\large Estudiante: A01797969\par}
    \vspace{0.5cm}
    {\large Ciencia y Analítica de Datos\par}
    \vfill
    {\large Octubre 4, 2025\par}
\end{titlepage}

\section{Introducción}

El cómputo cognitivo permite a las máquinas procesar y comprender información de manera similar a los humanos, transformando contenido no estructurado (imágenes, audio, video) en datos estructurados accionables. Este informe analiza tres plataformas líderes que ofrecen capacidades avanzadas de análisis cognitivo: AWS Rekognition, Microsoft Azure Face API, y Google Cloud Vision AI.

\section{Amazon Rekognition}

\subsection{Descripción}
\textbf{Proveedor:} Amazon Web Services (AWS). Servicio de análisis de imágenes y videos basado en \textit{deep learning}, sin requerir experiencia en ML.

\subsection{Funcionalidades y Aplicaciones}
Rekognition ofrece análisis facial (hasta 100 rostros/imagen) con detección de 8 emociones, rango de edad, género y atributos; comparación y verificación facial; detección de objetos y escenas; moderación de contenido; y procesamiento de video en tiempo real. Los datos extraídos incluyen coordenadas faciales, emociones con niveles de confianza, y atributos demográficos en formato JSON estructurado.

\textbf{Aplicaciones:} Aeropuertos (verificación de identidad), retail (análisis demográfico), NFL (identificación de jugadores), marketing (análisis emocional en \textit{focus groups}). Modelo ``pay-as-you-go'': \$1.00 USD/1,000 imágenes. \textit{Free tier}: 5,000 imágenes/mes (12 meses).

\section{Microsoft Azure Cognitive Services - Face API}

\subsection{Descripción}
\textbf{Proveedor:} Microsoft Corporation. Parte de Azure Cognitive Services, ofrece análisis facial avanzado con énfasis en privacidad y cumplimiento normativo.

\subsection{Funcionalidades y Aplicaciones}
Face API ofrece detección facial con 8 emociones, edad específica, género, postura de cabeza (\textit{yaw/pitch/roll}), atributos (vello, accesorios, maquillaje), verificación 1:1 e identificación 1:N, agrupación automática, y detección de viveza (\textit{anti-spoofing}) única entre las tres plataformas. Extrae datos en formato Python/JSON con \textit{faceId}, atributos detallados y métricas de calidad.

\textbf{Aplicaciones:} Salud (monitoreo emocional psiquiátrico), educación (\textit{engagement} en e-learning), RRHH (entrevistas virtuales), seguridad (búsqueda de personas). Microsoft requiere aprobación explícita para casos de vigilancia, cumpliendo GDPR. Costo: \$1.00 USD/1,000 transacciones. \textit{Free tier}: 30,000/mes.

\section{Google Cloud Vision AI y Video Intelligence API}

\subsection{Descripción}
\textbf{Proveedor:} Google Cloud Platform. Suite integrada que combina análisis de imágenes y videos mediante modelos pre-entrenados y personalizables.

\subsection{Funcionalidades y Aplicaciones}
\textbf{Vision AI} detecta etiquetas (objetos, ubicaciones, actividades), emociones faciales (alegría, tristeza, enojo, sorpresa), texto OCR (50+ idiomas), contenido explícito (SafeSearch), y logotipos comerciales. \textbf{Video Intelligence} ofrece segmentación temporal, \textit{tracking} de objetos \textit{frame-by-frame}, detección de personas, y transcripción de audio. Por políticas de equidad, Google eliminó detección de edad/género. Los datos se estructuran como \textit{faceAnnotations}, \textit{labelAnnotations} con coordenadas y probabilidades.

\textbf{Aplicaciones:} Getty Images (etiquetado masivo), retail (análisis de planogramas), manufactura (inspección de calidad), accesibilidad (descripciones automáticas). Costo: \$1.50/1,000 imágenes (detección facial), \$0.10/minuto (video). \textit{Free tier}: 1,000 unidades/mes por función.

\section{Comparación de Herramientas}

\begin{table}[h!]
\centering
\footnotesize
\begin{tabular}{@{}p{2.5cm}p{3.2cm}p{3.2cm}p{3.2cm}@{}}
\toprule
\textbf{Criterio} & \textbf{AWS Rekognition} & \textbf{Azure Face API} & \textbf{Google Vision} \\ \midrule
Precisión & Alta (94-98\%) & Muy alta (95-99\%) & Media-Alta \\
Edad/Género & Rangos, binario & Específica, binario & No disponible \\
Video & Nativo tiempo real & Integración adicional & API separada \\
Anti-spoofing & No & Sí & No \\
Casos principales & Seguridad, media & Salud, educación & Accesibilidad \\ \bottomrule
\end{tabular}
\caption{Comparación entre plataformas}
\end{table}

\section{Reflexión: Ventajas y Limitaciones}

\subsection{Ventajas}

\textbf{Democratización:} Permiten a organizaciones sin \textit{expertise} en ML implementar análisis sofisticados mediante APIs simples. \textbf{Escalabilidad:} Infraestructura cloud procesa desde imágenes individuales hasta millones de videos sin inversión en hardware. \textbf{Mejora continua:} Los modelos se reentrenan constantemente, mejorando \textit{accuracy} automáticamente. \textbf{Impacto multisectorial:} Desde diagnósticos médicos hasta experiencias retail personalizadas.

\subsection{Limitaciones y Consideraciones Éticas}

\textbf{Sesgos algorítmicos:} Estudios demuestran tasas de error más altas en personas de piel oscura, mujeres y grupos subrepresentados. \textbf{Privacidad:} El reconocimiento facial plantea dilemas sobre consentimiento y vigilancia masiva; varias ciudades han prohibido su uso policial. \textbf{Interpretación limitada:} Las emociones detectadas son expresiones faciales, no estados internos reales; la correlación varía culturalmente. \textbf{Dependencia de calidad:} Iluminación, ángulo y oclusiones afectan dramáticamente la precisión. \textbf{Regulación:} El AI Act (UE) y leyes de privacidad biométrica restringen estos usos. \textbf{Costos:} Aplicaciones de alto volumen generan costos significativos.

\subsection{Recomendaciones de Uso Responsable}

Es fundamental implementar \textbf{transparencia} (informar claramente el uso), \textbf{consentimiento explícito} (especialmente en contextos sensibles), \textbf{auditorías continuas} (evaluar desempeño en diferentes demografías), \textbf{uso complementario} (no tomar decisiones críticas basándose únicamente en estas herramientas), y \textbf{diseño ético} (considerar el impacto social desde el inicio).

\section{Conclusión}

Las tres plataformas analizadas representan el estado del arte en transformación de contenido multimedia a datos estructurados. AWS Rekognition destaca por video en tiempo real y completitud funcional; Azure Face API sobresale en privacidad y \textit{anti-spoofing}; Google Cloud Vision lidera en accesibilidad y ética proactiva.

La elección debe basarse en requisitos específicos: presupuesto, integración, privacidad y aplicación objetivo. Es imperativo adoptar marcos de IA responsable, priorizando equidad, transparencia y bienestar humano sobre optimización técnica. El futuro del cómputo cognitivo reside no solo en mejorar la precisión, sino en desarrollar sistemas justos, explicables y alineados con valores humanos fundamentales.

\begin{thebibliography}{9}
\bibitem{aws}
Amazon Web Services. (2024). \textit{Amazon Rekognition Developer Guide}. AWS Documentation.

\bibitem{azure}
Microsoft. (2024). \textit{Azure Cognitive Services Face API Documentation}. Microsoft Azure.

\bibitem{google}
Google Cloud. (2024). \textit{Vision AI and Video Intelligence API Documentation}. Google Cloud Platform.

\bibitem{bias}
Buolamwini, J., \& Gebru, T. (2018). Gender Shades: Intersectional Accuracy Disparities in Commercial Gender Classification. \textit{Proceedings of Machine Learning Research}, 81:1-15.

\bibitem{eu}
European Union. (2024). \textit{Artificial Intelligence Act}. Official Journal of the European Union.

\bibitem{nist}
National Institute of Standards and Technology. (2023). \textit{Face Recognition Vendor Test (FRVT) Results}. NIST.
\end{thebibliography}

\end{document}
